%=============================================================================
\chapter{Modelo de comportamiento del subsistema: Invitación de colaboradores}
\label{cap:reqSist}

En este capítulo se describen los casos de uso referentes a la invitación de colaborradores.

%---------------------------------------------------------

\begin{shaded}
		\textcolor{NavyBlue}{\Large\textbf{Elementos de un caso de uso}}
		\begin{itemize}
			\item \textbf{Resumen:} Descrpción textual del caso de uso
			\item \textbf{Actores:} Lista de los que  intervienen en el caso de uso.
			\item \textbf{Propósito:} Una breve descripción del objetivo que busca e actor al ejecutare el caso de uso.
			\item \textbf{Entradas:} Lista de los datos requeridos durante a ejecución del caso de uso.
			\item \textbf{Salidas:} Lista de los datos de salida que presetan al sistema durante la ejecuciń del caso de uso.
			\item \textbf{Precondiciones:} Descrpción de las operaciones o condiciones que se deben cumplir previamente para el caso de uso pueda ejecutarse correctamente.
			\item \textbf{Postcondiciones:} Lista de los cambios que ocurrirán en el sistema después de la ejecución del caso de uso y de las consecuencias del sistema.
			\item \textbf{Reglase de negocio:} Lista de las reglas que describen, limitan o controlan algún aspecto del negocio del caso de uso.
			\item \textbf{Errores:} Lista de los posibles errores que pueden surgir surante la ejecución del caso de uso.
			\item \textbf{Trayectoria:} Secuencia de los pasos que ejecutará el caso de uso
		\end{itemize}		
	\end{shaded}
\newpage

%----------------------------------------------------------------------
% CASOS DE USO
% \IUref{IUAdmPS}{Administrar Planta de Selección}
% \IUref{IUModPS}{Modificar Planta de Selección}
% \IUref{IUEliPS}{Eliminar Planta de Selección}

% 


% Copie este bloque por cada caso de uso:
%-------------------------------------- COMIENZA descripción del caso de uso.

%\begin{UseCase}[archivo de imágen]{UCX}{Nombre del Caso de uso}{
%--------------------------------------
	\begin{UseCase}{CU05}{Invitar un colaborador}{
		Permite a un líder de proyecto invitar a un colaborador a participar en un proyecto creado.
	}
		\UCitem{Actor}{\hyperlink{Líder de proyecto}{Líder de proyecto}}
		\UCitem{Propósito}{El líder de proyecto puede invitar un colaborador a participar en un proyecto.}
		\UCitem{Entradas}{\begin{itemize}
		\item Correo: Se ingresa desde el teclado
		\item Proyecto: Se selecciona de una lista
		\item Rol: Se selecciona de una lista.



		\end{itemize}
.}
		\UCitem{Salidas}{Colaborador invitado: Lo genera el sistema, Email de invitación: Lo genera el sistema 
}
		\UCitem{Destino}{Pantalla de colaboradores.}
		\UCitem{Precondiciones}{Haber iniciado sesión y tener un proyecto creado
}
		\UCitem{Postcondiciones}{El colaborador queda invitado al proyecto
 }
		\UCitem{Errores}{El usuario ya existe en el proyecto con ese rol.
}
		\UCitem{Observaciones}{}
	\end{UseCase}
%--------------------------------------
	\begin{UCtrayectoria}{Principal}
		\UCpaso[\UCactor] Da click en la opción mis invitaciones en el menú de colaboradores. 
		\UCpaso[\UCactor] Da click en la el botón invitar colaborador.
        \UCpaso[\UCactor]  Ingresa el correo del colaborador \label{crear}
       \UCpaso[\UCactor] Selecciona de una lista el proyecto.
       \UCpaso[\UCactor]  Selecciona de una lista el rol.
      \UCpaso[\UCactor]   Solicita invitar al colaborador mediante el botón invitar.  
      \UCpaso[\UCactor] Verifica que no se hayan omitido campos   \Trayref{A}
     \UCpaso[\UCactor] Verifica que exista una cuenta asociada al correo\Trayref{B}
       \UCpaso[\UCactor]Verifica que el colaborador no esté participando actualmente en el proyecto con el mismo rol\Trayref{C}
      \UCpaso  Envía un correo con un token y la información de la invitación
      
       \UCpaso   Muestra la pantalla de  colaboradores invitados.
	\end{UCtrayectoria}

%--------------------------------------		
		\begin{UCtrayectoriaA}{A}{ El actor no ingreso los datos requeridos}
			\UCpaso Muestra mensaje de falta de datos requeridos
			\UCpaso Continua en el paso \ref{crear} del \UCref{CU05}.
		\end{UCtrayectoriaA}
        
        
%--------------------------------------		
		\begin{UCtrayectoriaA}{B}{El correo ingresado no está asociado a una cuenta}
			\UCpaso Hace un preregistro del colaborador
			\UCpaso Envía un correo con la invitación al correo ingresado.
            \UCpaso[] Termina el caso de uso.
		\end{UCtrayectoriaA}
%--------------------------------------		
		\begin{UCtrayectoriaA}{C}{El correo ingresado no está asociado a una cuenta. }
			\UCpaso Muestra mensaje de que el colaborador ya se encuentra participando en el proyecto.
            \UCpaso[] Termina el caso de uso.
		\end{UCtrayectoriaA}
		
		
%-------------------------------------- TERMINA descripción del caso de uso.
% \IUref{IUAdmPS}{Administrar Planta de Selección}
% \IUref{IUModPS}{Modificar Planta de Selección}
% \IUref{IUEliPS}{Eliminar Planta de Selección}

% 


% Copie este bloque por cada caso de uso:
%-------------------------------------- COMIENZA descripción del caso de uso.

%\begin{UseCase}[archivo de imágen]{UCX}{Nombre del Caso de uso}{
%--------------------------------------
	\begin{UseCase}{CU06}{Responder una invitación}{
		Permite a un colaborador aceptar o declinar la invitación realizada por el líder de proyecto para participar en un proyecto.
	}
		\UCitem{Actor}{\hyperlink{Colaborador}{Colaborador}}
		\UCitem{Propósito}{Dar a conocer al líder de proyecto la respuesta sobre una invitación.}
		\UCitem{Entradas}{\begin{itemize}
		\item Token: Lo obtiene el sistema de la  URL.
		\item Correo: Lo obtiene el sistema
		\item Proyecto: Lo obtiene el sistema
		\item Rol: Lo obtiene el sistema
		\item Respuesta: Se selecciona con un botón
		\end{itemize}
.}
		\UCitem{Salidas}{Colaborador: Lo genera el sistema 
}
		\UCitem{Destino}{Pantalla de login.}
		\UCitem{Precondiciones}{Haber recibido una invitación}
		\UCitem{Postcondiciones}{El colaborador queda invitado al proyecto
 }
		\UCitem{Errores}{El token ha caducado.
}
		\UCitem{Observaciones}{}
	\end{UseCase}
%--------------------------------------
	\begin{UCtrayectoria}{Principal}
		\UCpaso[\UCactor]Da click en el link para responder la invitación.\label{enviar}
       \UCpaso Muestra la pantalla de inicio de sesión
      \UCpaso[\UCactor] Ingresa el correo electrónico y la contraseña de su cuenta  \Trayref{A} \label{responder}
     \UCpaso Verifica que el token no haya caducado
       \UCpaso   Muestra la pantalla de respuesta de invitación.
	\end{UCtrayectoria}

%--------------------------------------		
		\begin{UCtrayectoriaA}{A}{ El actor no ingreso los datos requeridos}
			\UCpaso Muestra mensaje de falta de datos requeridos
			\UCpaso Continua en el paso \ref{responder} del \UCref{CU06}.
		\end{UCtrayectoriaA}
        
        
%--------------------------------------		
		\begin{UCtrayectoriaA}{B}{el token ha caducado}
			\UCpaso Envía un correo al usuario con un nuevo token y los datos de la invitación
			\UCpaso Muestra mensaje de que la invitación ha caducado y el usuario debe revisar su correo.
			\UCpaso Continua en el paso \ref{enviar} del \UCref{CU06}.
		\end{UCtrayectoriaA}
		
		
%-------------------------------------- TERMINA descripción del caso de uso.
% \IUref{IUAdmPS}{Administrar Planta de Selección}
% \IUref{IUModPS}{Modificar Planta de Selección}
% \IUref{IUEliPS}{Eliminar Planta de Selección}

% 


% Copie este bloque por cada caso de uso:
%-------------------------------------- COMIENZA descripción del caso de uso.

%\begin{UseCase}[archivo de imágen]{UCX}{Nombre del Caso de uso}{
%--------------------------------------
	\begin{UseCase}{CU12}{Terminar tarea}{
		Permite a un colaborador dar por concluida una tarea asignada.
	}
		\UCitem{Actor}{\hyperlink{Colaborador.}{Colaborador.}}
		\UCitem{Propósito}{Dar por concluida una tarea asignada.}
		\UCitem{Entradas}{\begin{itemize}
		\item Usuario: lo obtiene el sistema
		\item Proyecto: lo obtiene el sistema
		\item Tarea iniciada: lo obtiene el sistema
		\item Entregable: lo ingresa el colaborador
		\end{itemize}
.}
		\UCitem{Salidas}{Tarea terminada: lo genera el sistema  
}
		\UCitem{Destino}{Tareas.}
		\UCitem{Precondiciones}{Tener una cuenta creada, tener una sesión iniciada, estar participando como colaborador en un proyecto, tener una tarea asignada y tener una tarea iniciada.
}
		\UCitem{Postcondiciones}{La tarea pasa a estado de terminada
 }
		\UCitem{Errores}{El archivo del entregable no es válido}
		\UCitem{Observaciones}{}
	\end{UseCase}
%--------------------------------------
	\begin{UCtrayectoria}{Principal}
      \UCpaso  Muestra la pantalla principal.
		\UCpaso[\UCactor]Da click en terminar tarea.
      \UCpaso  Muestra la pantalla de terminar tarea. \label{terminar}
		\UCpaso[\UCactor]Adjunta el entregable de la tarea.
		\UCpaso[\UCactor]Da click en aceptar.
      \UCpaso  Verifica que el entregable sea un archivo válido \Trayref{A}
      \UCpaso  Muestra la pantalla de tareas.
	\end{UCtrayectoria}
    
    %--------------------------------------		
		\begin{UCtrayectoriaA}{A}{ El actor ingresó un archivo no valido}
			\UCpaso Muestra mensaje de que el archivo ingresado es no valido
			\UCpaso Continua en el paso \ref{terminar} del \UCref{CU12}.
		\end{UCtrayectoriaA}

		
%-------------------------------------- TERMINA descripción del caso de uso.


% \IUref{IUAdmPS}{Administrar Planta de Selección}
% \IUref{IUModPS}{Modificar Planta de Selección}
% \IUref{IUEliPS}{Eliminar Planta de Selección}

% 


% Copie este bloque por cada caso de uso:
%-------------------------------------- COMIENZA descripción del caso de uso.

%\begin{UseCase}[archivo de imágen]{UCX}{Nombre del Caso de uso}{
%--------------------------------------
	\begin{UseCase}{CU13}{Ver mis invitaciones}{
		Permite al líder de proyecto ver el estado de las invitaciones que ha hecho a colaboradores para integrarse a sus proyectos.
	}
		\UCitem{Actor}{\hyperlink{Líder de Proyecto.}{Líder de Proyecto.}}
		\UCitem{Propósito}{Controlar las invitaciones que he realizado a mis colaboradores.}
		\UCitem{Entradas}{\begin{itemize}
		\item Usuario: lo obtiene el sistema
		\end{itemize}
.}
		\UCitem{Salidas}{Invitaciones: lo genera el sistema  
}
		\UCitem{Destino}{Invitaciones.}
		\UCitem{Precondiciones}{Tener una cuenta creada, tener una sesión iniciada, estar participando como líder en un proyecto
}
		\UCitem{Postcondiciones}{
 }
		\UCitem{Errores}{}
		\UCitem{Observaciones}{}
	\end{UseCase}
%--------------------------------------
	\begin{UCtrayectoria}{Principal}
      \UCpaso  Muestra la pantalla principal.
		\UCpaso[\UCactor]Da click en mis invitaciones.
      \UCpaso  Carga todas las invitaciones de los proyectos en los que es líder de proyecto.
      \UCpaso  Muestra la pantalla de invitaciones.
	\end{UCtrayectoria}

		
%-------------------------------------- TERMINA descripción del caso de uso.