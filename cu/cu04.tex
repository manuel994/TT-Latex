% \IUref{IUAdmPS}{Administrar Planta de Selección}
% \IUref{IUModPS}{Modificar Planta de Selección}
% \IUref{IUEliPS}{Eliminar Planta de Selección}

% 


% Copie este bloque por cada caso de uso:
%-------------------------------------- COMIENZA descripción del caso de uso.

%\begin{UseCase}[archivo de imágen]{UCX}{Nombre del Caso de uso}{
%--------------------------------------
	\begin{UseCase}{CU04}{Crear tarea}{
		Permite a un usuario la creación de una tarea en un proyecto ya creado.
	}
		\UCitem{Actor}{\hyperlink{Líder de proyecto}{Líder de proyecto}}
		\UCitem{Propósito}{Definir una tarea que pertenece a un proyecto.}
		\UCitem{Entradas}{\begin{itemize}
		\item Proyecto: Lo obtiene el sistema
\item Fecha de inicio: Se selecciona de un calendario
\item Fecha de térmico: Se selecciona de un calendario
\item Descripción: Se ingresa desde el teclado


		\end{itemize}
.}
		\UCitem{Salidas}{Tarea asignada: Lo genera el sistema.}
		\UCitem{Destino}{Pantalla de tareas del proyectos}
		\UCitem{Precondiciones}{Haber iniciado sesión y tener un proyecto creado
}
		\UCitem{Postcondiciones}{La tarea queda registrada en el sistema }
		\UCitem{Errores}{La fecha está fuera del rango del proyecto, la fecha de término es anterior a la fecha de inicio
}
		\UCitem{Observaciones}{}
	\end{UseCase}
%--------------------------------------
	\begin{UCtrayectoria}{Principal}
		\UCpaso[\UCactor] Da click en el menú de proyectos.
		\UCpaso[\UCactor] Da click en el ícono de tareas en la lista de proyectos.
        \UCpaso Muestra la pantalla de listado de tareas. 
       \UCpaso[\UCactor] Da click en el botón de crear tarea. 
       \UCpaso  Muestra la pantalla de crear tarea. \label{tarea}
      \UCpaso[\UCactor]  Ingresa el nombre, fecha de inicio, fecha de término y descripción.   \Trayref{A}
      \UCpaso[\UCactor]  Verifica que la fecha de inicio sea anterior a la fecha de fin.  \Trayref{B}
      \UCpaso[\UCactor]  Verifica que las fechas se encuentran dentro del rango del proyecto.    \Trayref{C}
       \UCpaso  Muestra la pantalla del listado de tareas.
	\end{UCtrayectoria}

%--------------------------------------		
		\begin{UCtrayectoriaA}{A}{ El actor no ingreso los datos requeridos}
			\UCpaso Muestra mensaje de falta de datos requeridos
			\UCpaso Continua en el paso \ref{tarea} del \UCref{CU04}.
		\end{UCtrayectoriaA}
%--------------------------------------		
		\begin{UCtrayectoriaA}{B}{ El actor ingresó una fecha de inicio inválida}
			\UCpaso Muestra mensaje de que las fechas no son válidas.
			\UCpaso Continua en el paso \ref{tarea} del \UCref{CU04}.
		\end{UCtrayectoriaA}
%--------------------------------------		
		\begin{UCtrayectoriaA}{C}{ El actor ingresó una fecha fuera del rango. }
			\UCpaso Muestra mensaje de que las fechas no son válidas.
			\UCpaso Continua en el paso \ref{tarea} del \UCref{CU04}.
		\end{UCtrayectoriaA}
		
		
%-------------------------------------- TERMINA descripción del caso de uso.