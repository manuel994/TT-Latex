%=========================================================
\chapter{Modelo de comportamiento del subsistema: Control de acceso}
\label{cap:reqSist}

En este capítulo se describen los casos de uso referentes al acceso a la aplicación, registro de cuenta y recuperacion de contraseñas.

%---------------------------------------------------------

\begin{shaded}
		\textcolor{NavyBlue}{\Large\textbf{Elementos de un caso de uso}}
		\begin{itemize}
			\item \textbf{Resumen:} Descrpción textual del caso de uso
			\item \textbf{Actores:} Lista de los que  intervienen en el caso de uso.
			\item \textbf{Propósito:} Una breve descripción del objetivo que busca e actor al ejecutare el caso de uso.
			\item \textbf{Entradas:} Lista de los datos requeridos durante a ejecución del caso de uso.
			\item \textbf{Salidas:} Lista de los datos de salida que presetan al sistema durante la ejecuciń del caso de uso.
			\item \textbf{Precondiciones:} Descrpción de las operaciones o condiciones que se deben cumplir previamente para el caso de uso pueda ejecutarse correctamente.
			\item \textbf{Postcondiciones:} Lista de los cambios que ocurrirán en el sistema después de la ejecución del caso de uso y de las consecuencias del sistema.
			\item \textbf{Reglase de negocio:} Lista de las reglas que describen, limitan o controlan algún aspecto del negocio del caso de uso.
			\item \textbf{Errores:} Lista de los posibles errores que pueden surgir surante la ejecución del caso de uso.
			\item \textbf{Trayectoria:} Secuencia de los pasos que ejecutará el caso de uso
		\end{itemize}		
	\end{shaded}
\newpage

%-------------------------------------------------------------------------
% \IUref{IUAdmPS}{Administrar Planta de Selección}
% \IUref{IUModPS}{Modificar Planta de Selección}
% \IUref{IUEliPS}{Eliminar Planta de Selección}

% 


% Copie este bloque por cada caso de uso:
%-------------------------------------- COMIENZA descripción del caso de uso.

%\begin{UseCase}[archivo de imágen]{UCX}{Nombre del Caso de uso}{
%--------------------------------------
	\begin{UseCase}{CU01}{Registro de usuario}{
		Permite a un usuario registrarse en el sistema.
	}
		\UCitem{Actor}{\hyperlink{Usuario}{Usuario}}
		\UCitem{Propósito}{Que el usuario pueda quedar registrado en el sistema.}
		\UCitem{Entradas}{\begin{itemize}
		\item Nombre: Se ingresa desde el teclado
		\item Apellidos: Se ingresa desde el teclado 
		\item Correo: Se ingresa desde el teclado 
		\item Contraseña: Se ingresa desde el teclado
		\end{itemize}
.}
		\UCitem{Salidas}{Usuario registrado: Lo genera el sistema.}
		\UCitem{Destino}{Pantalla de Login}
		\UCitem{Precondiciones}{El usuario no este registrado en el sistema.}
		\UCitem{Postcondiciones}{El usuario queda registrado en el sistema}
		\UCitem{Errores}{Ya hay un usuario registrado con ese correo electrónico}
		\UCitem{Observaciones}{}
	\end{UseCase}
%--------------------------------------
	\begin{UCtrayectoria}{Principal}
		\UCpaso[\UCactor] Da click en crear cuenta en la pantalla de inicio de sesión  .
        \UCpaso El sistema muestra la pantalla de registro.
		\UCpaso[\UCactor] Ingresa el nombre, apellidos, correo electrónico y la contraseña.
      \UCpaso[\UCactor]   Solicita registrar el usuario mediante el botón registrar.
		\UCpaso Verifica que no se hayan omitido campos  \Trayref{A}.
       \UCpaso  Verifica que no exista una cuenta asociada al correo 	\Trayref{B}.
       \UCpaso Muestra la pantalla principal
	\end{UCtrayectoria}

%--------------------------------------		
		\begin{UCtrayectoriaA}{A}{ El actor no ingreso los datos requeridos}
			\UCpaso Muestra mensaje de falta de datos requeridos
			\UCpaso Continua en el paso \ref{CU01 Registro de usuario} del \UCref{CU01}.
		\end{UCtrayectoriaA}
        
%--------------------------------------		
		\begin{UCtrayectoriaA}{B}{El actor ingreso un correo que ya está asociada a una cuenta}
			\UCpaso Muestra mensaje de que el correo ya está asociado a una cuenta
			\UCpaso Muestra la pantalla de inicio de sesión
            \UCpaso[] Termina el caso de uso.
		\end{UCtrayectoriaA}
		


		
		
		
%-------------------------------------- TERMINA descripción del caso de uso.
% \IUref{IUAdmPS}{Administrar Planta de Selección}
% \IUref{IUModPS}{Modificar Planta de Selección}
% \IUref{IUEliPS}{Eliminar Planta de Selección}

% 


% Copie este bloque por cada caso de uso:
%-------------------------------------- COMIENZA descripción del caso de uso.

%\begin{UseCase}[archivo de imágen]{UCX}{Nombre del Caso de uso}{
%--------------------------------------
	\begin{UseCase}{CU02}{Login}{
		Permite al colaborador y líder de proyecto autentificarse ante el sistema.
	}
		\UCitem{Actor}{\hyperlink{Colaborador, Líder de Proyecto}{Colaborador, Líder de Proyecto}}
		\UCitem{Propósito}{El colaborador o líder de proyecto puede ingresar al sistema.}
		\UCitem{Entradas}{\begin{itemize}
		\item Correo: Se ingresa desde el teclado
		\item Contraseña: Se ingresa desde el teclado
		\end{itemize}
.}
		\UCitem{Salidas}{Sesión creada: Lo genera el sistema.}
		\UCitem{Destino}{Pantalla principal}
		\UCitem{Precondiciones}{Ninguna}
		\UCitem{Postcondiciones}{El usuario ingresa al sistema}
		\UCitem{Errores}{El correo y/o contraseña no coinciden, el correo no está registrado
}
		\UCitem{Observaciones}{}
	\end{UseCase}
%--------------------------------------
	\begin{UCtrayectoria}{Principal}
		\UCpaso[\UCactor] Ingresa el correo electrónico y la contraseña.\label{Login}
		\UCpaso[\UCactor] Solicita iniciar sesión mediante el botón ingresar.
		\UCpaso Verifica que no se hayan omitido campos  \Trayref{A}.
       \UCpaso  Verifica que exista una cuenta asociada al correo	\Trayref{B}.
       \UCpaso  Verifica que la contraseña sea igual a la ingresada \Trayref{C}.
       \UCpaso Muestra la pantalla principal
	\end{UCtrayectoria}

%--------------------------------------		
		\begin{UCtrayectoriaA}{A}{ El actor no ingreso los datos requeridos}
			\UCpaso Muestra mensaje de falta de datos requeridos
			\UCpaso Continua en el paso \ref{Login} del \UCref{CU02}.
		\end{UCtrayectoriaA}
        
%--------------------------------------		
		\begin{UCtrayectoriaA}{B}{ El actor ingreso un correo que no está asociada a ninguna cuenta}
			\UCpaso Muestra mensaje de que el correo no está asociado a ninguna cuenta
			\UCpaso Muestra la pantalla de inicio de sesión
            \UCpaso[] Termina el caso de uso.
		\end{UCtrayectoriaA}
		        
%--------------------------------------		
		\begin{UCtrayectoriaA}{C}{ El actor ingreso una contraseña que no coincide con la contraseña registrada}
			\UCpaso Muestra mensaje de que la contraseña no es correcta
			\UCpaso Continua en el paso \ref{Login} del \UCref{CU02}.
		\end{UCtrayectoriaA}
		
		
%-------------------------------------- TERMINA descripción del caso de uso.
% \IUref{IUAdmPS}{Administrar Planta de Selección}
% \IUref{IUModPS}{Modificar Planta de Selección}
% \IUref{IUEliPS}{Eliminar Planta de Selección}

% 


% Copie este bloque por cada caso de uso:
%-------------------------------------- COMIENZA descripción del caso de uso.

%\begin{UseCase}[archivo de imágen]{UCX}{Nombre del Caso de uso}{
%--------------------------------------
	\begin{UseCase}{CU08}{Recuperar Contraseña}{
		Permite a un usuario recuperar su contraseña ingresando su correo electrónico.
	}
		\UCitem{Actor}{\hyperlink{Usuario.}{Usuario.}}
		\UCitem{Propósito}{Recuperar su cuenta.}
		\UCitem{Entradas}{\begin{itemize}
		\item Correo electrónico: se ingresa desde el teclado
		\item Nuevo contraseña: se ingresa desde el teclado
		\end{itemize}
.}
		\UCitem{Salidas}{Nueva contraseña asignada: Lo genera el sistema  
}
		\UCitem{Destino}{Login.}
		\UCitem{Precondiciones}{Tener una cuenta creada
}
		\UCitem{Postcondiciones}{La contraseña queda restablecida
 }
		\UCitem{Errores}{}
		\UCitem{Observaciones}{}
	\end{UseCase}
%--------------------------------------
	\begin{UCtrayectoria}{Principal}
      \UCpaso  Muestra el login.
		\UCpaso[\UCactor]Da click en el recuperar contraseña.
		\UCpaso[\UCactor]Ingresa el correo electrónico.\Trayref{A}
      \UCpaso  Envía un correo electrónico con un token.
		\UCpaso[\UCactor]Da click en el link para recuperar contraseña. \label{token}
       \UCpaso   Verifica que el token no haya caducado.\Trayref{B} 
       \UCpaso   Muestra la pantalla de recuperación de contraseña.
		\UCpaso[\UCactor]Ingresa su nueva contraseña.\Trayref{C} \label{password}
       \UCpaso   Muestra la pantalla de login.
	\end{UCtrayectoria}

%--------------------------------------		
		\begin{UCtrayectoriaA}{A}{ El actor ingresó un correo que se encuentre registrado en el sistema}
			\UCpaso Muestra mensaje de que se ha ingresado un correo que no se encuentra registrado
       \UCpaso   Muestra la pantalla de login.
            \UCpaso[] Termina el caso de uso.
		\end{UCtrayectoriaA}
%--------------------------------------		
		\begin{UCtrayectoriaA}{B}{El token ha caducado}
			\UCpaso Envía un correo al usuario con un nuevo token y el link de recuperación contraseña
			\UCpaso Muestra mensaje de que su token ha caducado y el usuario debe revisar su correo.
			\UCpaso Continua en el paso \ref{token} del \UCref{CU08}.
		\end{UCtrayectoriaA}
%--------------------------------------		
		\begin{UCtrayectoriaA}{C}{El actor ingresó mal la confirmación de su contraseña}
			\UCpaso Muestra mensaje de que la confirmación de contraseña no coincide.
			\UCpaso Continua en el paso \ref{password} del \UCref{CU08}.
		\end{UCtrayectoriaA}
        
		
%-------------------------------------- TERMINA descripción del caso de uso.
% \IUref{IUAdmPS}{Administrar Planta de Selección}
% \IUref{IUModPS}{Modificar Planta de Selección}
% \IUref{IUEliPS}{Eliminar Planta de Selección}

% 


% Copie este bloque por cada caso de uso:
%-------------------------------------- COMIENZA descripción del caso de uso.

%\begin{UseCase}[archivo de imágen]{UCX}{Nombre del Caso de uso}{
%--------------------------------------
	\begin{UseCase}{CU15}{Cambiar Contraseña}{
		Permite al usuario modificar su contraseña.
	}
		\UCitem{Actor}{\hyperlink{Usuario}{Usuario}}
		\UCitem{Propósito}{Que el usuario pueda modificar su contraseña despues de haber solicitado recuperarla}
		\UCitem{Entradas}{\begin{itemize}
		\item Contraseña: se ingresa desde teclado
		\item Confirmar contraseña: Se ingresa desde el teclado 
		\end{itemize}
.}
		\UCitem{Salidas}{Contraseña: contraseña actualizada}
		\UCitem{Destino}{Pantalla de Login}
		\UCitem{Precondiciones}{El usuario tenga un token generado para modificar su contraseña}
		\UCitem{Postcondiciones}{El usuario tendra su contraseña actualizada}
		\UCitem{Errores}{Las contraseñas no coinciden}
		\UCitem{Observaciones}{}
	\end{UseCase}
%--------------------------------------
	\begin{UCtrayectoria}{Principal}
		\UCpaso[\UCactor] Da click en el enlace que se encuentra en el correo electronico enviado.
        \UCpaso [\UCsist] accede a la pantalla \hyperref[fig:IU15]{IU15 Cambiar contraseña}. \label{item:CU15Item1}
		\UCpaso[\UCactor] ingresa los datos en los campos correspondientes. 
      \UCpaso[\UCactor]   solicita el cambio de su contraseña oprimiendo el botón aceptar.
		\UCpaso Verifica que no se hayan omitido campos  \Trayref{A}.
       \UCpaso  Verifica que no exista una cuenta asociada al correo 	\Trayref{B}.
       \UCpaso Muestra la pantalla principal
	\end{UCtrayectoria}

%--------------------------------------		
		\begin{UCtrayectoriaA}{A}{ El actor no ingreso los datos requeridos}
			\UCpaso Muestra mensaje de falta de datos requeridos
			\UCpaso Continua en el paso \ref{item:CU15Item1} del \UCref{CU15}.
		\end{UCtrayectoriaA}
        
%--------------------------------------		
		\begin{UCtrayectoriaA}{B}{la contraseñas introducidas no coinciden}
			\UCpaso Muestra el mensaje \MSGref{MSG12}{Las contraseñas no coinciden}
			\UCpaso Muestra en el campo de confirmacion de password
            \UCpaso[] Termina el caso de uso.
		\end{UCtrayectoriaA}
		


		
		
		
%-------------------------------------- TERMINA descripción del caso de uso.
